\chapter{Used Technology Analysis}
    According to the discussion in the previous chapter we have settled on docker containers running Ubuntu for providing the backbone of our virtualized networks. Now, once we are clear so as to what technology to employ we need to get down to the nitty gritty of implementing a full-fledged virtual network infrastructure. In order to do so we need to become acquainted with the Linux kernel.

    \section{Enter the Network Stack}
        A fascinating but rather messy image is \href{https://upload.wikimedia.org/wikipedia/commons/5/5b/Linux_kernel_map.png}{Linux's "Map"}. If we pay close attention we'll see how one of the columns is just devoted to networking. The software entities comprising this column is what we will refer to as \textit{Linux's Network Stack}\\

         The word stack is something that shows up time and time again in the area of networking. It helps us have a top-level view of how logical entities cooperate within a network. When we think about stacks we naturally begin to consider them in terms of the layers composing them with each layer tackling a simple task and offering services to the layer above whilst using those provided by the layer below. It's not going to be any different with Linux; we can think of its network stack as a huge "blob" of code which all network packets reaching a Linux-based system traverse. Thus, if we can alter how Linux processes packets or build "virtual" connections between different network stacks we would be capable of constructing a de-facto virtual network tailored to our needs.\\

         \paragraph{Naming Packets}
            Before we go on we need to shed some light on the naming we are going to use regarding the units of data exchanged through network links. Even though the term packet tremendously generic we feel it's not a wrong one to turn to in our case. When we wire up several network nodes together we are looking for full connectivity, that is, connectivity at the application level. Thus, we are not really that interested on what layer the "packet" is at, we don't really care if the packet is a \texttt{segment}, \texttt{datagram} or \texttt{frame}. In the case a need for more specific naming turns up we won't hesitate to use it but we prefer to keep the writing simple and avoiding getting bogged down with technicalities where we fill the benefit is not that obvious.

            A prime example of the above would be the use of the term \texttt{packet} instead of \texttt{link-layer frame} in the section's introduction. \texttt{Frame} is the correct term for referring to the data structure a \texttt{NIC} (Network Interface Card) hands to the kernel (albeit somewhat processed as the preamble and Frame Check Sequence of Ethernet frames are usually stripped from incoming frames by the \texttt{NIC} itself as seen \href{https://gitlab.com/wireshark/wireshark/-/wikis/Ethernet}{here}).