\subsubsection{\texttt{Constants.py}}
    In an effort to reduce the number of dependencies we decided to leverage \textit{ANSI Escape Codes} to colorize the different output messages. This module then defines the different \texttt{strings} controlling the colours.

    \paragraph{Imported Libraries}
        None as there are no external dependencies.

    \paragraph{Global Variables}
        \begin{enumerate}
            \item \textbf{\texttt{terminal\_escape\_sequences (dictionary/string/string)}:} \texttt{Dictionary} keyed by color names whose associated values are \texttt{ANSI Escape Sequences} changing the terminal's output color to the one specified. This solution is \textbf{not} to be considered portable. It \textbf{should} nonetheless work on any regular terminals supporting colors. We can assure it works with the following terminal emulators: \texttt{tilix}, \texttt{kitty} and \texttt{Visual Studio Code's Embedded Terminal Emulator}.
        \end{enumerate}