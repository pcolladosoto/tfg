% We prefer a oneside layout even though books
    % default to twoside
\documentclass[12 pt, oneside]{book}

% This package allows us to control the document wide
    % margins whilst also altering them at certain
    % pages
\usepackage[
    a4paper,
    left  = 2.5cm,
    right = 2.5cm
    % top = 2cm
    % bottom = 2cm
]{geometry}

% Allows using vertical bars in math mode
\usepackage{amsmath}

% Handles characters beyond ASCII nicely
\usepackage[utf8]{inputenc}

% Adds the Euro Symbol
\usepackage{eurosym}

% Table of Contents Control
    % Determines the ToC's Contents.
        % nottoc -> Do NOT include the ToC in the ToC itself.
        % notlof -> Do NOT include the LoF in the ToC.
        % notlot -> Do NOT include the LoT in the ToC.
    \usepackage[nottoc, notlof, notlot]{tocbibind}

    % Style the ToCs. Package documentation at:
        % http://mirrors.ibiblio.org/CTAN/macros/latex/contrib/tocloft/tocloft.pdf
        \usepackage{tocloft}

% Bibliography Stuff
    % See https://www.overleaf.com/learn/latex/Bibliography_management_in_LaTeX
    % Note the 'dashed = false' option prevents a repeated author's name to be replaced
        % by dashes (----) when both references share a page.
        \usepackage[backend = bibtex, style = ieee, dashed = flase]{biblatex}

        % Include our Bibliography Databases (i.e. *.bib files)
            \addbibresource{Bibliography/articles.bib}
            \addbibresource{Bibliography/figures.bib}
            \addbibresource{Bibliography/websites.bib}
            \addbibresource{Bibliography/manpages.bib}
            \addbibresource{Bibliography/books.bib}

% Figure Control
    % Handles the addition of figures
    \usepackage{graphicx}
    \graphicspath{{Figures/}}

    % Allows rotating floats such as figures and tables.
    \usepackage{rotating}

% Gantt Diagrams
\usepackage{pgfgantt}

% Allows the inclusion of PDF pages. We
    % use it to include pre-crafted covers.
\usepackage{pdfpages}

% Allows the easy addition of quotes
\usepackage{epigraph}

% Link Control
    \usepackage[
        pdfauthor  = {Pablo Collado Soto (pcolladosoto@gmx.com)}, % PDF Metadata
        pdftitle   = {Think of Something Witty...},               % PDF Metadata
        pdfborder  = {0 0 0},                                     % Remove the ugly border
        colorlinks = true,                                        % Let there be color
        citecolor  = black,                                       % Make citations appear normal (i.e black)
        linkcolor  = black,                                       % The same for links (i.e table of contents)
        urlcolor   = cyan                                         % Color the web links. (cyan is fancy for blueish...)
    ]{hyperref}

% Code Listings
    % See https://www.overleaf.com/learn/latex/Code_listing
    \usepackage{listings, lstautogobble}
    \usepackage{xcolor}

    % Color definitions to be used in the listings
    \definecolor{codegreen}{rgb}{0,0.6,0}
    \definecolor{codegray}{rgb}{0.5,0.5,0.5}
    \definecolor{codepurple}{rgb}{0.58,0,0.82}
    \definecolor{backcolour}{rgb}{0.95,0.95,0.92}

    % Defines an style for the listings
    \lstdefinestyle{lststyle}{
        % backgroundcolor = \color{backcolour},
        commentstyle = \color{codegreen},
        keywordstyle = \color{magenta},
        numberstyle = \tiny\color{codegray},
        stringstyle = \color{codepurple},
        basicstyle = \ttfamily\footnotesize, % Consider using \scriptsize!
        breakatwhitespace = false,
        breaklines = true,
        captionpos = b,
        keepspaces = true,
        numbers = left,
        numbersep = 5pt,
        showspaces = false,
        showstringspaces = false,
        showtabs = false,
        tabsize = 2,
        autogobble = true
    }

    % This style defines a left margin to visually center CSV files
    \lstdefinestyle{csvs}{
        % backgroundcolor = \color{backcolour},
        commentstyle = \color{codegreen},
        keywordstyle = \color{magenta},
        numberstyle = \tiny\color{codegray},
        stringstyle = \color{codepurple},
        basicstyle = \ttfamily\footnotesize, % Consider using \scriptsize!
        breakatwhitespace = false,
        breaklines = true,
        captionpos = b,
        keepspaces = true,
        numbers = left,
        numbersep = 5pt,
        showspaces = false,
        showstringspaces = false,
        showtabs = false,
        tabsize = 2,
        autogobble = true,
        xleftmargin = 3.2cm,
        % xrightmargin=1cm
    }

    \lstset{style = lststyle}
