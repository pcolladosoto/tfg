\chapter{Resumen}
    \thispagestyle{empty}

    El objetivo del Trabajo de Fin de Grado propuesto es el desarrollo de un sistema \textit{software} que automatice el despliegue y gestión de redes puramente virtuales para su uso como entorno de pruebas. El trabajo pertenece al ámbito del proyecto \textit{CloudWall}\footnote{Este trabajo ha sido parcialmente financiado por el proyecto PID2019-104855RB-I00/AEI/10.13039/501100011033 del Ministerio de Ciencia e Innovación.} del \textit{Departamento de Automática} de la \textit{Universidad de Alcalá}, cuya meta primordial es el desarrollo de una infraestructura de resiliencia de red adapatada a las necesidades de los sistemas informáticos del sistema sanitario para así incrementar su capacidad de prevención y reacción ante ataques cibernéticos. Nuestro trabajo busca servir como mecanismo de validación para las ténicas obtenidas del proyecto \textit{CloudWall}.\\

    El uso de contenedores de \textit{docker} como nodos de red virtuales junto a las posibilidades brindadas por el \textit{kernel} de \textit{linux} ofrecen una gran cantidad de flexibilidad que hemos respetado y proporcionado al usuario. La lógica que implementa las funciones de control de red se ha escrito íntegramente en \textit{python3}. Asimismo, se ha desarrollado una prueba de concepto para demostrar la adecuación del resultado obtenido para el uso que se le pretendía dar en un principio. Además, dadas las tecnologías empleadas, las aplicaciones que se le pueden dar a este trabajo son más amplios de lo inicialmente requerido. Consideramos que una de las aplicaciones futuras más prometedoras es su posible uso como una herramienta de apoyo a labores docentes.\\

    \textbf{Palabras clave:} \textit{docker} \cite{bib:docker}, \textit{iproute2} \cite{bib:man-ip}, \textit{namespaces} \cite{bib:man-namespaces} \textit{python3} \cite{bib:python}.
